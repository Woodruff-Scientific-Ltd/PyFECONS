

\section{Background to the Fusion Concept}

\subsection{Mirrors}
This section presents an overview of key literature in the development of magnetic mirror fusion reactors. This includes their early development, early confinement challenges, how the concept was sidelined in favour of tokamak development, and how there has been a recent return to the concept, with exciting new prospects. 

\begin{itemize}
    \item In "Fusion Research in Open-Ended Systems", 1969, by T.K. Fowler, the focus is on low-beta fusion research in open-ended magnetic geometries, known as magnetic mirrors \cite{fowler1969fusion}. This work provides an overview of magnetic mirror research, discussing methods, achievements, and rationale. It introduces magnetic mirrors for the confinement of hot fusion plasmas in configurations where plasma is held between a pair of coils, creating stronger magnetic fields near the coils. The configuration allows for the escape of certain velocity particles, known as the mirror loss cone, and contrasts with closed systems like toroidal configurations.

 \item In "Concept for a High-Power-Density Mirror Fusion Reactor", 1973, R.F. Post, T.K. Fowler, J. Killeen, and A.A. Mirin propose a new mirror-type fusion reactor design \cite{post1973concept}. Utilizing advancements in neutral-beam technology, this concept improves plasma stability, power density, and the energy gain factor (Q) over previous designs.

 \item "Mirror Reactor Studies", 1976, by R.W. Moir and colleagues at Lawrence Livermore Laboratory introduces a fusion mirror reactor with 150-keV neutral-beam injectors, providing over 1 GW of continuous power \cite{moir1976mirror}. The reactor features a three-stage modularized Venetian blind plasma direct converter with a 59\% efficiency and a novel method for removing the lune-shaped blanket, addressing the challenge of low Q and high recirculating power.

 \item In "Improved Tandem Mirror Fusion Reactor", 1979, D.E. Baldwin and B.G. Logan discuss barrier potentials in tandem mirror reactors to reduce ion energy and density requirements \cite{baldwin1979improved}. This innovation, involving raising the plug-electron temperature, marks a significant advancement in magnetic mirror fusion technology.

 \item In "Magnetic Mirror Fusion - Status and Prospects", 1980, Post compares the concept of magnetic mirror fusion to closed magnetic systems like tokamaks \cite{post1980magnetic}. It is explained that mirror confinement, an open system, features magnetic field lines extending beyond the confinement area, contrasting with closed systems where these lines are contained. The importance of the magnetic mirror effect for longitudinal confinement is highlighted, with emphasis on the necessity for particles to have sufficient rotational energy for effective trapping. The "loss cone" in velocity space, influenced by the mirror ratio and particle energy ratio, is also described. The paper notes that the imbalance in electron and ion loss rates is compensated by the development of a positive ambipolar potential, affecting the overall plasma loss rate. It is concluded that higher temperatures, which reduce ion-ion collision rates and boost fusion rates, could improve the fusion power balance in mirror confinement systems.

 \item "Experimental Progress in Magnetic-Mirror Fusion Research", 1981, by Thomas C. Simonen emphasizes the progression from basic mirror cells to advanced designs in magnetic mirror systems like tandem mirrors \cite{simonen1981experimental}. It discusses improvements in plasma containment and the reduction of end losses, along with the technical aspects of magnetic mirror confinement.

 \item "The Engineering of Magnetic Fusion Reactors", 1983, by Robert W. Conn delves into the construction of magnetic fusion reactors using the tandem mirror method, highlighting challenges and techniques in maintaining plasma temperature and energy capture \cite{conn1983engineering}.


 \item "Mirror-Based Fusion: Some Possible New Directions", 1999, by Richard F. Post highlights innovative approaches in magnetic mirror fusion - offering several benefits, including suppression of interchange-type MHD instability modes due to positive field-line curvature, and the ability to introduce and control particle beams for improved fueling and ash removal \cite{post1999mirror}. Additionally, Post suggests a shift towards low-Q open-ended systems for their relaxed confinement requirements and better-understood physics. This approach brings potential for more efficient, smaller, and less expensive fusion systems and opens possibilities for using alternative fusion fuels.
 
 \item In "Fifty Years of Magnetic Fusion Research (1958-2008): Brief Historical Overview and Discussion of Future Trends", 2010, by Laila A. El-Guebaly, magnetic mirror fusion is discussed as one of the original magnetic confinement concepts alongside tokamak, stellarator, and pinch \cite{el2010fifty}. Tandem Mirror (TM) research, a variation of magnetic mirror fusion, began in the 1950s. By the mid-1970s, the TM concept was proposed, offering advantages like high beta (30–70\%), no driven plasma current, and the potential for direct conversion of charged particle power into electricity. The TM design consists of a long central cell with solenoidal coils terminated by end mirror cells and direct conversion systems. In the 1980s, major TM experimental facilities were built in the US, including MFTF-B and TMX-U, and conceptual power plant designs like WITAMIR, MARS, MINIMARS, and Ra were developed. However, by the late 1980s, the US Department of Energy shifted focus away from TMs to concepts with seemingly fewer challenges, like tokamaks. Despite the early promise, the TM concept has seen limited growth in recent years, with existing magnetic traps like the gas-dynamic trap and multi-plug trap at the Budker Institute of Nuclear Physics not being TMs.

 \item "Modern magnetic mirrors and their fusion prospects", 2010, by A.V. Burdakov et al. present advancements in magnetic mirror technology, focusing on the Gas Dynamic Trap (GDT) and the Multi-Mirror Trap from the Budker Institute of Nuclear Physics \cite{burdakov2010modern}. These developments represent key steps in making magnetic mirror technology viable for practical fusion applications.

  

  \item In "Concept of Fusion Reactor Based on Multiple-Mirror Trap", 2011, A.V. Burdakov and colleagues focus on advancements in multiple-mirror confinement for fusion reactors, using the GOL-3 device in Novosibirsk as a case study \cite{burdakov2011concept}. It reports significant improvements in plasma behavior, facilitated by new collective phenomena and enhanced plasma heating methods. These advancements, including effective heat transport control and magnetohydrodynamic stabilization, have improved the viability of multiple-mirror confinement systems for practical fusion reactor applications. The study also provides a perspective on the development of large-scale fusion devices using this technology.

 \item In "Gas-dynamic Trap: An Overview of the Concept and Experimental Results", 2013, A.A. Ivanov and V.V. Prikhodko explore the Gas Dynamic Trap (GDT) - a type of magnetic mirror reactor \cite{ivanov2013gas}. This reactor is characterized by its long distance between mirrors and high mirror ratio, enabling effective plasma confinement. The GDT's design facilitates plasma stability and offers promising applications in neutron source development for fusion materials and hybrid fusion-fission systems. The paper also reviews experimental results from GDT trials, providing insights into its potential in future fusion research.

  \item In "Progress in Mirror-Based Fusion Neutron Source Development", 2015, by the Budker Institute of Nuclear Physics  advancements in developing a 14 MeV neutron source for fusion material studies are detailed \cite{anikeev2015progress}. This source, based on the Gas Dynamic Trap (GDT), a specialized magnetic mirror system, has achieved stable confinement of hot-ion plasmas with relative pressures exceeding 0.5 and elevated the electron temperature to 0.9 keV. These achievements, including the highest electron temperature reached in axisymmetric open mirror traps, position the GDT-based neutron source as a promising tool for material testing and fusion-fission hybrid systems.

 \item In "Magnetic-confinement Fusion", 2016, by J. Ongena et al., the concept and history of magnetic mirrors in fusion research is presented \cite{ongena2016magnetic}.  The paper describes magnetic mirrors as working by increasing the magnetic field strength at both ends of a confinement region, using additional coils to create a 'magnetic bottle'. This design repels plasma particles with an identical pole to the strong end magnetic fields, trapping them within the bottle. Despite this innovative approach, achieving effective confinement in mirror machines proved challenging, primarily due to instabilities caused by end losses. Particles with velocities mainly along the magnetic field line would not be stopped by the end mirrors and escape, leading to a non-Maxwellian velocity distribution and instabilities. This issue ultimately led to the abandonment of the magnetic-mirror approach in 1986 when the U.S. decided not to operate the Mirror Fusion Test Facility B (MFTF-B).

 \item In "Encouraging Results and New Ideas for Fusion in Linear Traps", 2018, by P.A. Bagryansky and colleagues reviews recent advancements in linear trap technologies \cite{bagryansky2019encouraging}. This study highlights the efficiency of multiple-mirror confinement with gas-dynamic traps and introduces the helical-mirror variant for improved confinement of rotating plasmas, laying a foundation for future fusion energy exploration and development.


 \item In "Fusion by beam ions in a low collisionality, high mirror ratio magnetic mirror", 2022, Egedal et al.'s study introduces the Fast Beam Ion Solver (FBIS) model to improve the understanding of ion behavior in magnetic mirror geometries, contributing to the design of WHAM++, a cost-effective magnetic mirror device \cite{egedal2022fusion}.

\end{itemize}
\label{MIF_lit_review}

\subsection{Novatron concept}

The Novatron Fusion Group, a start-up based in Stockholm, is working on an innovative approach to magnetic fusion energy. Their concept, an evolution from conventional mirror confinement methods, aims to solve the poor confinement issues that have hindered traditional designs. The design is characterized by an axisymmetric, high mirror-ratio plasma that strategically aims to suppress interchange instabilities through its unique curvature.\\

A key aspect of Novatron's design is the integration of a mirror-cusp magnet topology, which marries the features of a classic magnetic mirror with those of a biconic cusp. This design ensures a normal magnetic field akin to that of a traditional magnetic mirror, while also adopting the favorable curvature features of a biconic cusp. The axisymmetric design is crucial in minimizing mid-plane leakage and neoclassical transport, common problems in classical cusp configurations.\\

Addressing the technical challenges in plasma containment, such as interchange instabilities and axial losses, Novatron has developed specific strategies. These include vortex stabilization, which involves rotating the plasma around a central axis, and anchor cells that create regions of favorable curvature to counteract the central cell's unfavorable curvature. The design also incorporates line tying, using conducting plates to facilitate electron flow from regions of unfavorable to favorable curvature.\\

The design further tackles axial losses, particularly those stemming from drift cyclotron loss-cone (DCLC) modes. Techniques such as filling ambipolar poles and employing centrifugal potentials for ion trapping are key components of Novatron's strategy against these instabilities. The design's approach to managing DCLC modes involves a nuanced understanding of plasma dynamics and magnetic field manipulation.\\

In terms of magnet design evolution, Novatron emphasizes high magnetic field symmetry. This is achieved through a three-coil system that shapes the magnetic field to enhance confinement and stability. This system marks a significant improvement over previous models by addressing the issue of unfavorable pressure gradient plasma, a challenge observed in earlier experiments like the RFC-XX-M.\\

As Novatron prepares for its proof-of-principle experimental facility, with initial experiments scheduled for late 2023, the fusion community observes with interest. The goal is to progress towards a 1.5 GWe reactor, Novatron 4, by the end of the 2020s. While some details, such as fuel choice and energy conversion methods, remain unspecified, Novatron's fusion concept represents a thoughtful and technically sound step towards realizing efficient and sustainable fusion power.