\textbf{Foreword} \\ 
 
In the last several years, the developer of this model has maintained an active interest in the costing analysis for fusion energy systems, collaborating with many in the field, and publishing a series of papers on this topic starting in 2011 with basic community consensus on the "Path to Market" for fusion energy systems, outlining the constraints placed on any energy technology as it moves towards commercialization in terms of timeline and available capital, and market demand for smaller, more modular units. This was followed with an analysis with a specific embodiment of a compact modular fusion power core that could be developed privately, working in the context of an IAEA Coordinated Research Project on compact fusion neutron sources and developing a costing model based on the ARIES methodology \cite{waganer2006design}. In 2017, additional support from ARPA-E was gained to further develop the costing model in collaboration with Bechtel. For that costing study, development considered 4 of the concepts that were supported by ARPA-E under the ALPHA program, and worked with PIs in each organization to develop basic radial builds of the 4 fusion power cores, based on prior art, and then performed cost analysis with the Balance of Plant and tritium systems analysis provided by Bechtel, but based predominantly on historical studies. In 2018 the developer organized a workshop for the IAEA on private fusion enterprises, and the commercialization path for fusion, with contributions from most major private fusion institutions and provided a summary of the cost modeling underway.\\
 
In 2019 ARPA-E revisited the 2017 costing analysis in the light of new costing paradigms developed by Ingersoll and Foss at Lucid Catalyst addressing every assumption in the Cost Accounting Structure (CAS), calculating an updated \$/kW for most CAS outside of the fusion power core (drawing in particular on the DOE NETL report). The 2019 ARPA-E study revisited the analysis using the code developed for the four concepts, touching on most of the recommendations of the prior study and obtaining review by leading experts in the field. The costing was revisited in 2020 to extend to all ARPA-E supported fusion concepts, and the final version was started in 2023 to be released publicly. This model captures the most recent thinking and brings together current understanding of best costing methodologies uncovered in the course of the ARPA-E work.\\

\textbf{Disclaimer} \\

The objective of this study is to present a methodological framework for understanding the cost structure of a fusion power plant. To this end, the model integrates a comprehensive—though not exhaustive or fully accurate—set of cost categories, including pre-construction costs, direct costs, indirect service costs, owner's costs, supplementary costs, and financial costs.

The outputs generated by the model are not intended to represent absolute or realistic values of the Levelized Cost of Electricity (LCOE) for the simulated plant. Instead, the results should be interpreted solely as relative indicators, suitable for comparison with other similar modelling exercises and for identifying the primary cost drivers within the plant configuration analysed.

The cost elements currently included in the model are incomplete and do not fully capture the cost structure of a future commercial fusion power plant. Consequently, the absolute LCOE values derived from this model should not be considered accurate or representative of real-world deployment.

The model will be progressively refined as additional data become available and as technological and economic parameters are better understood, with the long-term objective of producing LCOE estimates that more closely reflect the realities of commercial fusion power plants.\\
\newpage