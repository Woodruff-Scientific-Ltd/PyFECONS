\ifsafetycosts
\section{Safety and hazard cost basis}

This appendix summarises the main safety and hazard considerations that
are explicitly translated into cost categories in this report, based on
the underlying safety analysis and literature (for example the fusion
safety assessments of Piet et al.\ (1982)\cite{Piet1982Potential} and
the ITER engineering design and safety reports\cite{iter2001final,
baylor2019shattered}).

\subsection{Radioactive release of tritium from storage (CAS 11)}

For inertial fusion concepts with significant tritium throughput and
dust inventories, land and land-rights costs can be influenced by
potential radioactive releases from storage and processing systems. As
in the MFE case, the land and land-rights cost \(C110000\) is
decomposed into:
\begin{itemize}
  \item \(C110100\): baseline land cost as a function of power, and
  \item \(C110200\): additional land and mitigation cost associated with
        tritium and dust release.
\end{itemize}

The safety-driven term \(C110200\) is evaluated using a polynomial fit
to site-boundary calculations for tritium and dust releases as a
function of stack height and mobilisable inventory, following
Lukacs \& Williams (2020). The required site radius for meeting
emergency reference levels is converted to a land area and cost using
representative US farm real-estate values\cite{USDA2022LandValues}.

\subsection{Plant licensing (CAS 13)}

As for MFE, plant licensing costs for IFE concepts are captured in Cost
Category 13 – Plant Licensing (\(C130000\)) and depend on the selected
region. The same literature base is used (US NRC options for fusion
licensing, UK REPPIR 2019, and related policy work\cite{White2021Regulation,
NRCFusion2023,UKREPPIR2019,UKFusion2022}), but the numerical values can
be adjusted independently for IFE-specific deployments.

\subsection{Third-party liability insurance (CAS 78)}

The same LSA-based approach is used to estimate third-party liability
insurance for IFE as for MFE\cite{Holdren1991Safety,Miller2003ARIESST,
NRCInsurance2024}. Fusion reactors with LSA values of 2--3 are assumed
to have substantially lower risk than conventional fission plants, and
the annual premium is scaled accordingly from Price-Anderson fission
benchmarks. The resulting premium appears as \(C780000\) in Cost
Category 78 – Taxes and Insurance and is also included in the annualised
O\&M totals.

\subsection{Scope and limitations}

The safety-related cost components described above provide a
first-order, literature-based costing of key safety-driven systems and
land-use requirements for IFE. They are intended to:
\begin{itemize}
  \item make explicit where and how safety and hazard mitigation
        considerations affect the CAS structure for IFE,
  \item allow simple sensitivity studies on tritium inventory, region,
        and LSA level, and
  \item complement, rather than replace, a full, site-specific safety
        case and detailed engineering design.
\end{itemize}
\fi

