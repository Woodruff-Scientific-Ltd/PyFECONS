% copied from 2023ARPAECosting/MFE_Menard_01082024_1/CAS220109_MFE.tex on 2024-02-15

\subsubsection*{Cost Category 22.01.09: Direct Energy Convertor}

A direct energy converter for fusion applications is a groundbreaking advancement in energy technology, notable for its remarkable efficiency of up to 70\% in converting plasma energy to electricity. Key components include the Expander Tank, which reduces the power loading from diverted plasma, and the Expander Coils, essential for directing plasma towards electrodes that convert the energy.  Consists of:

\begin{itemize}
    \item Cost Category 22.01.09.01 Expander Tank. Converts surface power loading to tolerable levels from the diverted plasma to incidents surfaces.
    \item Cost Category 22.01.09.02 Expander Coils.  Provides the magnetic funnel to direct plasma towards surfaces.
    \item Cost Category 22.01.09.03 Neutron Trap.  Provides shielding of components in the direct energy converter from neutron damage.
    \item Cost Category 22.01.09.04 Vacuum System (separate from the chamber vaccum system).  Provides pumping on the direct energy converter.
    \item Cost Category 22.01.09.05 Collector Electrode system.  Allows charged particles to be captured on electrode surfaces for energy to be directly converted.
    \item Cost Category 22.01.09.06 Thermal Management.  Cooling system for electrodes - either passively radiatively or actively with a liquid or gaseos coolant.  Inlcudes heat rejection system.
    \item Cost Category 22.01.09.07 Electrical subsystem.  Provides the coupling to the balance of plant systems, listed in Cost Category 24. Electric Plant Equipment.
\end{itemize}


The collector can accrue damage from sputtering caused by the ion fluence. For tungsten ribbons, this erosion rate is estimated to be 0.02mm/year, resulting in a low likelihood of having to replace grid elements.\\

Using additively manufactured ribbons with a simple water coolant channel, a heat flux of 2MW/m$^3$ is acceptable, reducing the overall volume and thus the cost of the subsystem by a factor of 60\% compared with a 1MW/m$^3$ flux radiatively-cooled system.\\


\begin{table}[ht]
    \centering
    \resizebox{0.6\linewidth}{!}{%
    \begin{tabular}{|l|r|}
    \hline
    \textbf{Item} & \textbf{Cost (M USD)} \\ \hline
    Expander tank & EXPANDER_TANK \\ \hline
    Expander Coil and Neutron Trap Coil & EXPANDER_COIL_AND_NEUTRON_TRAP_COIL \\ \hline
    Convertor gate valve & CONVERTOR_GATE_VALVE \\ \hline
    Neutron Trap Shielding & NEUTRON_TRAP_SHIELDING \\ \hline
    Vacuum system & VACUUM_SYSTEM \\ \hline
    Grid system & GRID_SYSTEM \\ \hline
    Heat collection system & HEAT_COLLECTION_SYSTEM \\ \hline
    Electrical & ELECTRICAL_EQUIPMENT \\ \hline
    Cost per unit & COST_PER_UNIT \\ \hline
    Total unit cost & TOTAL_DEUNIT_COST \\ \hline
    Engineering (15\%) & ENGINEERING_15_PERCENT \\ \hline
    Contingency (15\%) & CONTINGENCY_15_PERCENT \\ \hline
    Total facility cost & C220109 \\ \hline
    \end{tabular}}
    \caption{Cost Table}
    \label{tab:cost-table}
\end{table}

An embodiment of a direct energy converter consists broadly of an expander and collector, comprising a grid of ribbons arranged in a `venetian blind' configuration.
% \cite{post1970mirror}. % TODO - implement citation
Various expander geometries have been proposed.
% \cite{post1970mirror}. % TODO - implement citation
A conical configuration is often selected, due to the lower cost below 800 keV, as well as the ability to maintain the individual modules.
% \cite{barr1974preliminary}. % TODO - implement citation
This direct energy conversion has an efficiency of $\sim$ 60 - 70\%.
% \cite{moir1973venetian}. % TODO - implement citation
Thus, with the converter structure, the total cost is \$ C220109 M.
