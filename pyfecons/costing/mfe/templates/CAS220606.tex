\ifsafetycosts
\subsubsection*{Cost Category 22.06.06: Remote Handling System}

Hot cell operations in fusion power plants require sophisticated remote handling systems to manage highly activated components while minimizing worker exposure. These systems enable remote inspection, maintenance, and replacement of in-vessel and hot cell components that cannot be accessed directly due to high radiation levels. They are therefore a critical part of the overall safety strategy for managing radioactive materials and ensuring regulatory compliance.

One reference implementation is the MASCOT telemanipulator system deployed at the JET fusion facility, which demonstrates the complexity of modern remote maintenance solutions, including long-reach booms, dexterous manipulators with haptic feedback, and advanced control interfaces \cite{Hamilton2001Mascot}. For commercial-scale power plants, the ITER divertor remote handling system provides a relevant cost basis: a contract worth approximately 40 million EUR (about 55 million USD) was awarded for its design, manufacture, delivery, installation, commissioning, and associated movers and tooling \cite{IterRemoteHandlingWNN}.

In this costing framework, we adopt a fixed cost of 55 million USD for the remote handling system, applied to magnetic confinement fusion machines (MFE) when safety and hazard mitigation costs are enabled.

\begin{table}[ht]
    \centering
    \begin{tabular}{lr}
        \hline
        \textbf{Item} & \textbf{Cost (M USD)} \\ \hline
        Remote Handling System (Hot Cell and In-Vessel Maintenance) & C220606 \\ \hline
    \end{tabular}
    \caption{Cost basis for the remote handling system supporting hot cell and in-vessel maintenance operations.}
    \label{tab:cas220606-cost}
\end{table}

The total cost allocated to Cost Category 22.06.06 is \$ C220606 M.
\fi

