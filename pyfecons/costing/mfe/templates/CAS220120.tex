\ifsafetycosts
\subsubsection*{Cost Category 22.01.20: Disruption Mitigation System}

Plasma disruptions in tokamak fusion reactors represent a significant safety and operational concern. These events involve rapid loss of plasma confinement, resulting in intense particle and heat fluxes on the first wall, as well as electromagnetic forces on structural components. To mitigate these effects, all tokamak devices must incorporate disruption mitigation systems.

The chosen concept for the ITER disruption mitigation system utilizes shattered pellet injection (SPI), a technology developed at Oak Ridge National Laboratory and first demonstrated at the DIII-D tokamak \cite{baylor2019shattered}. This technique involves injecting large quantities of neon and deuterium into the plasma as solid ice pellets, which are shattered into small pieces before entering the vacuum vessel. The largest pellets, measuring 28 mm in diameter, must be fired simultaneously in multiple quantities to mitigate the worst-case runaway electron beam. A system layout allows for the injection of up to 32 pellets from equatorial ports at three different toroidal locations and three pellets from different upper ports.

The cost basis for this system is derived from the ITER Engineering Design Activities Final Report \cite{iter2001final}, which estimated the total cost of the pellet injection system at 5.66433 kIUA (kilo ITER Unit of Account), where 1 IUA = USD 1,000 (January 1989 value). This base cost has been adjusted to current year dollars using the specified inflation rate.

\begin{table}[ht]
    \centering
    \begin{tabular}{lr}
        \hline
        \textbf{Item} & \textbf{Cost (M USD)} \\ \hline
        Disruption Mitigation System (Shattered Pellet Injection) & C220120 \\ \hline
    \end{tabular}
    \caption{Cost for the disruption mitigation system.}
    \label{tab:cas220120-cost}
\end{table}

The total cost for the disruption mitigation system is \$ C220120 M.
\fi