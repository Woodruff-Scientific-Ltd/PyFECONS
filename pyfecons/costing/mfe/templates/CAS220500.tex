% copied from 2023ARPAECosting/MFE/Latex/Originals/CAS220500_DT.tex

\subsubsection{Cost Category 22.05: Fuel Handling and Storage}

The fuel handling and storage system in fusion reactors is responsible for the online processing of fuel isotopes, encompassing extraction, recovery, purification, preparation, and storage. Separate from fuel injection, this system processes materials from various sources including liquid breeders, chamber gases, and tritium-bearing streams, ensuring safe tritium levels in power core and heat transfer fluids under normal and emergency conditions.  \\

While most equipment is commercially available, heightened reliability and integrity standards, especially for tritium-containing materials, lead to higher costs. For example, atmospheric tritium recovery systems are costly due to the need for rapid and efficient tritium concentration reduction. This account's functionality and requirements are well-understood from existing and developing fusion facilities, with revisions from previous versions. Cost increases are expected due to higher reliability demands for power plant applications, but this may be offset by learning curve effects in subsequent plant constructions. The recommended Life-Cycle Analysis (LSA) factors for all accounts are 0.85 (LSA1) and 0.94 (LSA 2), reflecting these considerations.\\

  Consists of: 

\begin{itemize}
\item Cost Category 22.05.01 Chamber exhaust gas handling and processing equipment (H\&P).  Equipment for filtering, treating, and safely managing exhaust gases from the reactor chamber.
\item Cost Category 22.05.02 Purge and cover gas H\&P. Systems used for handling and processing gases that protect fuel and reactor components from contamination.
\item Cost Category 22.05.03 Primary coolant stream H\&P. Equipment involved in the handling and processing of the primary coolant stream, essential for maintaining coolant integrity.
\item Cost Category 22.05.04 Purification and isotope separation. Systems for the purification of reactor materials and the separation of isotopes, vital for fuel quality and waste management.
\item Cost Category 22.05.05 Tritium, deuterium and DT storage. Facilities for storing tritium, deuterium, and DT mixtures, which are integral to nuclear fusion reactions. 
\item Cost Category 22.05.06 Atmospheric tritium recovery. Systems designed to recover tritium from the atmosphere within the reactor facility, crucial for environmental protection and resource conservation
\end{itemize}

Various previous systems and cost analyses are available as extrapolation points, in particular For this report, ITER is used as the benchmark, then linearly scaled by $P_E$, scaled for inflation, and a learning curve credit of LEARNING_CURVE_CREDIT is applied, equating to a factor of LEARNING_TENTH_OF_A_KIND for a nth-of-a-kind system.  The resulting total cost is C220500 M USD. A breakdown of costs for each susbsystem for ITER and this concept can be found in table \ref{tab:fuel}.



\begin{table}
    \centering
    \begin{tabular}{lcc}
    \hline
        Cost Category & ITER Costs (M USD) & Scaled Costs (M USD)\\
        \hline
       22.05.01  & C2205010ITER & C220501\\
       22.05.02  & C2205020ITER & C220502\\
       22.05.03  & C2205030ITER & C220503\\
       22.05.04  & C2205040ITER & C220504\\
       22.05.05  & C2205050ITER & C220505\\
       22.05.06  & C2205060ITER & C220506\\
       22.05.00  & C22050ITER   & C220500\\
       \hline
    \end{tabular}
    \caption{Breakdown of fuel handling and storage subsystems for ITER and the concept presented in this report.}
    \label{tab:fuel}
\end{table}
