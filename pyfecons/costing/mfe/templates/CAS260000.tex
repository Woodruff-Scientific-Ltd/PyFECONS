\subsection{Cost Category 26: Heat Rejection} 

This cost scales primarily with the difference between the thermal power and the gross electric power (heat rejection). Scales per Delene as overall system with spares allowance.  Subsystems include water intake, circulating water systems, cooling towers, and other systems which reject heat to the atmosphere. The cooling towers provide the ultimate heat sink for the condenser cooling and service water systems.  For this cost study, the cooling towers are assumed to be mechanical draft (fans) rectangular cell-type towers.  On-site power (standby power) will be available to sustain operation of the plant following a loss of off-site power. The site is assumed to be located near a surface water source for makeup water needs of the plant.  The power plant is assumed to be located in a warm/humid climate at a nominal plant elevation of 100 feet above sea level.  
\begin{itemize}
    \item Cost Category 26.01.00 Structures: Includes structures for the make-up water and intake, the circulating
water-pump house, the make-up water pretreatment building, and the cooling towers.
\item Cost Category 26.02.00 Mechanical Equipment: Includes the heat rejection mechanical equipment such as circulating water pumps, piping, valves, mechanical draft cooling towers, water treatment plant, intake water pumps, screens, and filters. Excludes condensers and natural draft towers.
\end{itemize}

Cost basis: NETL baseline NGCC costs. Exhibit 5-7, Cost Category 9: Cooling System. Bare erected cost excluding engineering, construction mgmt, and contingency \$28 million in source table for NGCC / 263 MW electric capacity in source table for NGCC = \$107/kW to give a total of C260000 M USD.









