\ifsafetycosts
\section{Safety and hazard cost basis}

This appendix summarises the main safety and hazard considerations that are
explicitly translated into cost categories in this report, based on the
underlying safety analysis and literature (for example the fusion safety
assessments of Piet et al.\ (1982)\cite{Piet1982Potential} and the ITER
engineering design and safety reports\cite{iter2001final,baylor2019shattered}).

\subsection{Radioactive release of tritium from storage (CAS 11)}

The land and land-rights cost \(C110000\) is decomposed into a baseline
site land component \(C110100\) and a safety-driven component
\(C110200\) linked to potential radioactive release of tritium from
storage and dust inventories. The safety component is evaluated using a
polynomial fit to site-boundary calculations for tritium and dust
releases as a function of stack height and mobilisable inventory, based
on the work of Lukacs \& Williams (2020) and related dispersion
studies. The required site radius for meeting emergency reference levels
for sheltering/evacuation is converted to a land area and cost using
representative US farm real-estate values\cite{USDA2022LandValues}.

In the CAS structure this appears as:
\begin{itemize}
  \item \(C110100\): baseline land cost as a function of neutron and
        fusion power (as in the original 2022 update), and
  \item \(C110200\): additional land and mitigation cost associated
        with tritium and dust release from storage and processing
        systems.
\end{itemize}
The total land and land rights cost is \(C110000 = C110100 + C110200\)
when safety and hazard mitigation costs are included.

\subsection{Plasma disruption effects and disruption mitigation (CAS 22.01.20)}

For the tokamak configuration considered here, plasma disruptions can
impose severe thermal and electromagnetic loads on the first wall and
surrounding components\cite{Piet1982Potential}. Modern designs therefore
include a dedicated disruption mitigation system (DMS) based on
shattered pellet injection, as developed for ITER and demonstrated on
devices such as DIII-D\cite{baylor2019shattered,iter2001final}.

In this costing model the capital cost of a full disruption mitigation
system is captured as a dedicated Heat Island Component:
\begin{itemize}
  \item Cost Category 22.01.20: Disruption mitigation system
        (\(C220120\)).
\end{itemize}
The magnitude of \(C220120\) is based on ITER DMS cost estimates,
adjusted for units and inflation, and is only applied for magnetic
fusion energy (MFE) machines.

\subsection{Hot-cell radioactivity and remote handling (CAS 22.06.06)}

Activated in-vessel components (blankets, divertor, first wall, etc.)
must be handled and maintained using remote systems in a hot-cell
environment. Experience from JET and ITER shows that such systems
require substantial specialised equipment and tooling, including
multi-degree-of-freedom manipulators and dedicated movers and casks
\cite{hamilton2001development,contractforiter}.

The cost of this capability is represented as:
\begin{itemize}
  \item Cost Category 22.06.06: Remote handling equipment
        (\(C220606\)),
\end{itemize}
which is then rolled up into the broader Other Plant Equipment
subtotals. This term is included for MFE concepts where in-vessel
remote handling is required.

\subsection{Plant licensing (CAS 13)}

Plant licensing costs depend on the regulatory framework and region. For
fusion, recent work by regulators and policy bodies (e.g.\ the US NRC
options for licensing fusion energy systems and the UK REPPIR 2019
framework) indicates that fusion facilities will be treated differently
from traditional fission plants, but will still incur non-negligible
licensing and regulatory costs\cite{White2021Regulation,NRCFusion2023,
UKREPPIR2019,UKFusion2022}.

These costs are captured in:
\begin{itemize}
  \item Cost Category 13: Plant Licensing (\(C130000\)),
\end{itemize}
with a base cost derived from fission capital cost studies
\cite{WorldNuclearEconomics} and a safety-driven regional add-on that
depends on the selected region (US, UK, or unspecified). The add-on
aggregates representative fees for licensing, radiological emergency
preparedness, and environmental permitting in each jurisdiction.

\subsection{Third-party liability insurance (CAS 78)}

Third-party liability insurance for fusion plants is expected to be
significantly lower than for fission reactors, reflecting the reduced
inventory of long-lived radionuclides and the different accident
phenomenology\cite{Holdren1991Safety,UKFusion2022}. To obtain a
transparent and scalable estimate, the model starts from US nuclear
liability insurance premiums under the Price-Anderson framework (as
reported by the American Nuclear Insurers and the NRC
\cite{NRCInsurance2024}) and scales them using the Level of Safety
Assurance (LSA) concept from the ARIES studies\cite{Miller2003ARIESST}.

For the MFE and IFE cases considered here, LSA values of 2--3 are
assumed, corresponding to a risk fraction of roughly 19--23\% of a
conventional fission plant. The annual premium is then:
\[
  \text{Premium}_\text{fusion} \approx f_\text{LSA} \times
  \text{Premium}_\text{fission},
\]
where \(f_\text{LSA}\) is the LSA-based risk factor. The resulting
annual premium appears in:
\begin{itemize}
  \item Cost Category 78: Taxes and Insurance (\(C780000\)),
\end{itemize}
and is also included in the annualised O\&M totals.

\subsection{Scope and limitations}

The safety-related cost components described above are deliberately
simple and transparent. They are intended to:
\begin{itemize}
  \item make explicit where and how safety and hazard mitigation
        considerations affect the CAS structure,
  \item allow rapid sensitivity studies (e.g.\ varying tritium
        inventory, LSA level, or region), and
  \item provide a clear bridge between detailed safety analyses and the
        high-level cost accounting in this report.
\end{itemize}
They do \emph{not} replace a full safety case or detailed site-specific
engineering design; rather, they provide a first-order, literature-based
costing of key safety-driven systems and land-use requirements.
\fi

