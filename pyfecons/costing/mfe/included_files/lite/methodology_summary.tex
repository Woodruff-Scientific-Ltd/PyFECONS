\section{Summary of Methodology of Fusion Costing Framework}

This cost report is based on a comprehensive methodology that draws on three major reports that summarize cost categories for power plants: the methodology developed by the IAEA, the GENIV Economics Modeling Working Group (G4EMWG), and Geoffrey Rothwell's book 'Economics of Nuclear Power'. The framework uses well-documented and contemporary cost basis methodologies to provide a systematic approach to fusion power plant cost estimation.\\

The cost categories in this framework depart from prior fusion costing studies by organizing costs into:
\begin{itemize}
    \item \textbf{Category 10}: Pre-construction costs
    \item \textbf{Category 20}: Construction costs (direct costs)
    \item \textbf{Categories 30-60}: Indirect costs (formerly 91-98), including indirect service costs, owner's costs, supplementary costs, and financial costs
    \item \textbf{Category 70}: Operation and Maintenance costs
    \item \textbf{Category 80}: Fuel costs
    \item \textbf{Category 90}: Decommissioning costs
\end{itemize}

The framework adopts the subcategory descriptions from the GENIV methodology, adapted specifically for fusion systems. Fusion systems differ significantly from fission systems in the heat island, fuel cycle, handling, and replacement schedule of major cost components.\\

Key foundational documents include:
\begin{itemize}
    \item \textbf{G4EMWG Guidelines} \cite{EMWGOGIIF2007}: Comprehensive guide for economic modeling of Generation IV nuclear energy systems, providing detailed cost estimation structure
    \item \textbf{Rothwell's "Economics of Future Nuclear Power"} \cite{Rothwell2015}: Analysis of economic factors, cost estimation, risk assessment, and capital cost determination for nuclear power projects
    \item \textbf{IAEA Tecdoc TRS396} \cite{Meyer2000}: Guidelines for economic evaluation in the nuclear power sector, including methods for calculating Levelized Discounted Electricity Generation Costs
    \item \textbf{ARIES Cost Account Documentation} \cite{Waganer2013}: Historical economic basis for fusion costing analyses, documenting cost data from various fusion design studies
    \item \textbf{NETL Cost and Performance Baseline} \cite{JamesCorrespondingAuthor2019}: Independent assessment of cost and performance for fossil energy power systems, used for comparative cost basis
\end{itemize}

The methodology emphasizes providing recent cost basis information for all cost categories, with references to support transparency and reproducibility. The framework is designed to evolve as additional data becomes available and technological parameters are better understood.\\

\newpage
