\subsection{Cost Category 80: Annualized Fuel Cost (AFC)}

%\subsection{Annual Fuel Cost} 
The fuel cost, C$_{F}$, is calculated as follows.  The unit cost of deuterium as D2 is 3,700 \$/kg; deuterium contributes negligibly to the COE of a fusion power plant. In the long run, the power plant is self-sufficient in terms of tritium fuel production because of the breeding capability of the blanket so that no specific tritium-fuel charge is reported. It should be recognized, however, that there is a significant cost for tritium in the direct cost of the D-T fueled fusion reactor, represented in Cost Category 22.5 Fuel Handling and Storage. Cost of the lead lithium materials is included in Cost Category 27 Special Materials.\\

Consists of:  
\begin{verbatim} 
m_D = 3.342*10^(-27) # (kg)
u_D = 2175 #Where u_D ($/kg) = 2175 ($/kg) 
C_F = N_mod * P_NRL * 1e6 * 3600 * 8760 * u_D * m_D * p_a / (17.58 * 1.6021e-13)
\end{verbatim} 

Total annual fuel costs are \$ C800000 M per year.

\subsubsection*{Cost Category 81 – Refueling Operations}
This Cost Category includes incremental costs associated with refueling operations.

\subsubsection*{Cost Category 84 – Fuel}
This Cost Category includes annualized costs associated with the fuel cycle.

\subsubsection*{Cost Category 86 – Processing Charges}
This Cost Category includes storage and processing if fuel is brought in from offsite.

\subsubsection*{Cost Category 87 – Special Nuclear Materials}
This Cost Category covers materials such as heavy water, sodium, lead, helium, or other energy transfer mediums that are required on an annual basis. It includes costs associated with disposal or treatment if necessary. 

\subsubsection*{Cost Category 89 – Contingency on Annualized Fuel Costs}
This Cost Category includes an assessment of additional cost necessary to achieve the desired confidence level for the annualized fuel costs not to be exceeded.
