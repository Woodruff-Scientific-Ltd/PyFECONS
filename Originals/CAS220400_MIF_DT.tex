
\subsubsection{Cost Category 22.4: Radioactive Waste Treatment} 
Consists of: radioactive waste treatment and disposal facility: 22.4.1 liquid waste processing and equioment; 22.4.2 gaseous wastes and off-gas processing system; and 22.4.3 solid waste processing equipment. Costs are scaled relative to thermal power, per ARIES-ST and escalted relative to 1992 \$ \cite{DEL90b}. The resultant total cost is C220400 M USD. \\

The objective of radioactive waste treatment in a fusion reactor includes:
\begin{itemize}
\item Purifying the Liquid Breeder or Heat Transfer Media: This involves maintaining the concentration of radioactive products, in this case PbLi, below specific safety limits.

\item Classifying Radioactive Materials for Recycling or Disposal: According to El-Guebaly et al. in “Goals, Challenges, and Successes of Managing Fusion Active Materials,” radioactive materials must be prepared for either recycling, clearance, or proper disposal in designated off-site repositories.

\item On-Site Treatment and Management: The on-site system is designed for basic processing and management, not for extensive processing. It does not serve as the primary system for online separation of Deuterium and Tritium fuel isotopes, which is handled by the Fuel Handling and Storage system (Account 22.5). Nonetheless, cost-effectively recovered tritium, such as from detritiation processes, is reintegrated into the Fuel Handling and Storage system for further refinement.

\item Equipment and Location: The processing equipment is likely located in the Hot Cell building. This equipment encompasses remote handling tools, storage tanks, pumps, piping, valves, heat exchangers, heaters, condensers, gas strippers, compressors, chemical reactors, evaporators, ion exchange subsystems, filters, traps, and separators.

\end{itemize}

It's important to note that fusion reactors produce less long-lived radioactive waste compared to fission reactors. The IAEA notes that fusion reactor waste is primarily composed of activated structural materials, and the volume of high-level waste is significantly less \cite{girard2008summary}. This makes waste management in fusion reactors potentially more manageable and less hazardous over the long term. The development of low-activation structural materials remains a key enabling technology for fusion implementation and public acceptance.



