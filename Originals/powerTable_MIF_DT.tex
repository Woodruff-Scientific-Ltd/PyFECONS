%needs packages \usepackage{graphicx}   \usepackage{float} 
\newpage

\begin{table}[h]								
\centering
\resizebox{0.9\linewidth}{!}{%
\begin{tabular}{cp{6.5cm}ccc}								
\hline								
\textbf{Account}	&	\textbf{Power Account Description}	&	\textbf{Parameter }	&	\textbf{Power}	&	\textbf{Units} \\
\hline								
1	&	\textbf{Output power}	&		&		&	\\
1.2	&	Alpha Power	&	$P_{{\alpha}}$	&	P_ALPHA	&	MW \\
1.3	&	Neutron Fusion Power	&	$P_{{neutrons}}$	&	P_NEUTRON	&	MW \\
1.4	&	Neutron Energy Multiplier	&	$M_N = P_{{TH}}/P_{{fusion}}$	&	M_N	&	\\
1.5	&	Pumping power capture efficiency	&	$\eta_{{pump}}$	&	eta_p	&	\\
1.6	&	Thermal Power	&	$P_{{TH}}$	&	PTH	&	MW \\
1.7	&	Thermal conversion efficiency	&	$\eta_{{TH}}$	&	eta_th	&	\\
1.8	&	Total (Gross) Electric Power	&	$P_{{ET}}$	&	PET	&	MW \\
1.9	&	Lost Power	&	$P_{{Lost}}$	&	PLOSS	&	MW \\
\hline								
2	&	\textbf{Recirculating power}	&		&		&	\\
2.1	&	Power into coils 	&	$P_{{coils}} = P_{{cs}} + P_{{m}}+ P_{{d}}+ P_{compress}-P_{rec}$	&	P_coils	&	MW \\
2.1.1 & Power into CS coil & $P_{\text{cs}}$ & P_tf & MW \\
2.1.2 & Power into Mirror coils & $P_{\text{m}_e}$ & P_e & MW \\
2.1.2 & Power into Divertor coils & $P_{\text{d}_e}$ & P_d & MW \\
2.2	&	Primary Coolant Pumping Power Fraction	&	$f_{{pump}}$	&	f_p&	\\
2.2.1	&	Primary Coolant Pumping Power	&	$P_{{pump}} = f_{{pump}} \cdot P_{{ET}}$	&	P_p	&	MW \\
2.3	&	Subsystem and Control Fraction	&	$f_{{sub}}$	&	f_sub	&	\\
2.3.1	&	Subsystem and Control Power	&	$P_{{sub}} + P_{{control}} = f_{{sub}} \cdot P_{{ET}}$	&	P_sub	&	MW \\
2.4	&	Auxiliary systems	&	$P_{{aux}} = P_{{t}} + P_{{h}}$	&	P_aux	&	MW \\
2.4.1	&	Tritium Systems	&	$P_{{t}}$	&	P_t	&	MW \\
2.4.2	&	Housekeeping power	&	$P_{{h}}$	&	P_h	&	MW \\
2.5	&	Cooling systems	&	$P_{{cool}} = P_{{m}_c} + P_{{cs}_c}+P_{{d}_c}$	&	P_cool	&	MW \\
2.5.1 & Mirror coil cooling & $P_{\text{m}_c}$ & P_tf_c & MW \\
2.5.2 & CS coil cooling & $P_{\text{cs}_c}$ & P_e_c & MW \\
2.5.2 & Divertor coil cooling & $P_{\text{d}_c}$ & P_d_c & MW \\
2.5.3	&	Compression system cooling	&	$P_{{compress}_c}$	&	P_compress_c	&	MW \\
2.5.4	&	Cryo vacuum pumping	&	$P_{{p}_c}$	&	P_p_c	&	MW \\
2.6	&	Input power	&	$P_{{IN}}$	&	P_IN	&	MW \\
2.6.1	&	Compression system wall plug efficiency	&	$\eta_{{compress}}$	&	eta_compress	&	\\
2.6.3	&	Power into compression system	&	$P_{{compress}}$	&	P_compress	&	MW \\
2.6.4	&	Efficiency of compression recovery	&	$\eta_{{rec}}$	&	eta_compress_r	&	 \\
2.6.5	&	Power recovered from compression	&	$P_{{compress}_r}$	&	P_rec	&	MW \\
\hline								
3	&	\textbf{Outputs}	&		&		&	\\
3.1	&	Scientific Q	&	$Q = P_{{fusion}}/P_{{IN}}$	&	Q_S	&	\\
3.2	&	Engineering Q	&	$Q_{{E}}$	&	Q_E	&	\\
3.3	&	Recirculating power fraction	&	$\epsilon = 1/Q_{{E}}$	&	epsilon_	&	\\
3.4	&	Output Power (Net Electric Power)	&	$P_{{E}} = (1 - \epsilon) \cdot P_{{ET}}$	&	PE	&	MW \\
\hline								
\end{tabular}	
}
\caption{Power balance for MIF with magnetic mirrors concept.}
\end{table}	

%With certain terms assumed as nominal values and a target recirculating power fraction taken to be $\epsilon$ = 0.20, a fusion power plant design space can be defined in terms of thermal conversion efficiency, $\eta_{TH}$ , and input efficiency, $\eta_{IN}$, in order to display isoquants of the necessary value of fusion 200.00, Q.  This design space is generalized; power terms do not appear explicitly, but once any specific power is selected, all other power terms are determined.  The highest values of $\eta_{IN}$ may be impractical and values of $\eta_{TH}$ approaching 0.60 require advanced Brayton-cycle technology.  The fusion power-plant design space is shown in Fig. \ref{fig:pbal}. With advanced thermal-conversion features \cite{Dabiri1989}, including a Brayton cycle, $\eta_{TH}$ can approach 0.60, or even 0.65 - 0.75 using a direct energy converter. A conventional Rankine thermal cycle should yield $\\eta_{TH}$ in the range 0.40 - 0.45.
\\\\ 

%\begin{figure}[h!] 
%\centering
%\includegraphics[scale=0.6]{Figure3.eps}
%\caption{Power-Balance design space.}
%\label{fig:pbal}
%\end{figure}


%Fig \ref{fig:pbal} depicts the design space in terms of $\eta_{IN}$ versus $\eta_{TH}$ for the indicated fixed parameters with recirculating power fraction less that 20 percent. 

							
