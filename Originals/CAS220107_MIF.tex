\subsubsection*{Cost Category 22.1.7: Power Supplies} 

This category covers an extensive range of power supply systems critical for operation. This includes High-Voltage Power Supplies for plasma heating, Pulsed Power Supply Systems for plasma initiation, and Superconducting Coil Power Supplies for magnetic confinement. Auxiliary Heating Power Supplies support additional plasma heating, while Cooling and Cryogenic Systems maintain optimal temperatures for superconducting components. 


\begin{itemize}
    \item Cost Category 22.07.01 \textbf{High-Voltage Power Supplies}: Essential for driving heating systems like neutral beam injectors, which heat the plasma to the necessary temperatures for fusion.
    \item Cost Category 22.07.02 \textbf{Pulsed Power Supply Systems}: These provide high-power electrical pulses required for plasma initiation and shaping.
    \item Cost Category 22.07.03 \textbf{Coil Power Supplies}: These are crucial for large superconducting magnets, and copper coils for equilibrium control.
    \item Cost Category 22.07.04 \textbf{Auxiliary Heating Power Supplies}: These are used for additional plasma heating systems like ion cyclotron resonance heating (ICRH) and electron cyclotron resonance heating (ECRH).
    \item Cost Category 22.07.05 \textbf{Cooling and Cryogenic Systems}: To maintain the necessary low temperatures for the superconducting magnets and other components.
    \item Cost Category 22.07.06 \textbf{Control and Instrumentation Power Supplies}: For the myriad of sensors, diagnostics, and control systems involved in monitoring and controlling the fusion reactor.
    \item Cost Category 22.07.07 \textbf{Power Conversion Systems for Plasma Heating}: Such as gyrotrons for ECRH and klystrons or tetrodes for ICRH.
    \item Cost Category 22.07.08 \textbf{Safety and Interlock Systems}: To ensure safe operations, especially during abnormal events or emergencies, these systems can cut power or initiate shutdown sequences.
    \item Cost Category  22.07.09 \textbf{Diagnostics and Monitoring Systems Power Supply}: For real-time monitoring of plasma and reactor conditions, requiring reliable and clean power sources.
\end{itemize}

%need to escalate this cost relative to the 2011 data from ITER:
%https://docs.google.com/spreadsheets/d/1MHiUnJ580Vxzbb7P5V9vQH0PnGHsuDlx/edit?usp=sharing&ouid=106425516412438351916&rtpof=true&sd=true
Cost basis is taken from ITER for a 500MW fusion system, scaling the 269.6kIUA (1kIUA is 2MUSD), giving 539M USD for a 500MW fusion system. The total cost is C220107 M USD for a PNRL GW fusion system.


%\begin{figure}[h!]
%    \centering
%    \includegraphics[scale=0.4]{LEF_Numx.eps}
%    \caption{Lifetime extension factor vs ratio of applied to rated voltage}
%    \label{fig:1}
%\end{figure}

%\begin{figure}[h!]
%    \centering
%    \includegraphics[scale=0.4]{LOB_Numx.eps}
%    \caption{Lifetime of bank in years, at 1Hz vs ratio of applied to rated voltage}
%    \label{fig:2}
%\end{figure}


%\begin{figure}[h!]
%\centering
%\includegraphics[scale=0.4]{BCF_Numx.eps}
%\caption{Bank cost factor vs ratio of applied to rated voltage}
%\label{fig:3} 
%\end{figure}

